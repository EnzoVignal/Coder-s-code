%% start of file `template-zh.tex'.
%% Copyright 2006-2013 Xavier Danaux (xdanaux@gmail.com).
%
% This work may be distributed and/or modified under the
% conditions of the LaTeX Project Public License version 1.3c,
% available at http://www.latex-project.org/lppl/.


\documentclass[12pt,a4paper,sans]{moderncv}   % possible options include font size ('10pt', '11pt' and '12pt'), paper size ('a4paper', 'letterpaper', 'a5paper', 'legalpaper', 'executivepaper' and 'landscape') and font family ('sans' and 'roman')

% moderncv 主题
\moderncvstyle{banking}                        % 选项参数是 ‘casual’, ‘classic’, ‘oldstyle’ 和 ’banking’
\moderncvcolor{blue}                          % 选项参数是 ‘blue’ (默认)、‘orange’、‘green’、‘red’、‘purple’ 和 ‘grey’
%\nopagenumbers{}                             % 消除注释以取消自动页码生成功能

% 字符编码
\usepackage[utf8]{inputenc}                   % 替换你正在使用的编码
\usepackage{CJKutf8}

\usepackage{tikz}

\usepackage[lf]{venturis} %% lf option gives lining figures as default;
			  %% remove option to get oldstyle figures as default
%\renewcommand*\familydefault{\sfdefault} %% Only if the base font of the document is to be sans serif
\usepackage[T1]{fontenc}

% 调整页面
\usepackage[scale=0.85]{geometry}
%\setlength{\hintscolumnwidth}{3cm}           % 如果你希望改变日期栏的宽度

% 个人信息
\name{夏永锋}{}
\title{个人简历}                     % 可选项、如不需要可删除本行
\phone[mobile]{15921584916}              % 可选项、如不需要可删除本行
\email{youngsterxiayf@gmail.com}                    % 可选项、如不需要可删除本行
\homepage{blog.xiayf.cn}            % 可选项、如不需要可删除本行
\extrainfo{\url{github.com/youngsterxyf}}
%----------------------------------------------------------------------------------
%            内容
%----------------------------------------------------------------------------------
\begin{document}
\begin{CJK}{UTF8}{gkai}                       % 详情参阅CJK文件包
\maketitle


\section{工作背景}

\cventry{2014.05 -- 至今}{高级Web研发工程师}{百度}{上海}{}{}
\begin{itemize}
	\item {\color{blue}\href{http://ce.baidu.com}{百度云观测 \tikz \draw[->, thick] (0,0) cos(0.1,0.2);}}
	\begin{itemize}
		\item 简介:为中小站点提供可用性、访问速度、安全等方面的监测服务
		\item 个人职责:Web后端功能开发,开放API设计与实现,微信服务号开发与维护 等
		\item 相关技术:PHP、Yii、Yaf、Python、Go、Git、Gitlab
	\end{itemize}
\end{itemize}

\cventry{2013.04 -- 2014.05}{运营开发工程师}{腾讯}{上海}{}{}
\begin{itemize}
	\item 内部运营平台
	\begin{itemize}
		\item 简介:针对易迅业务的线上服务器进行监控系统的收集、展示、告警 等
		\item 个人职责:
		\begin{itemize}
			\item 服务器信息配置模块为所有告警模块、数据图形化展示提供基础数据
			\item 服务器归类模块方便分类展示、检索服务器
			\item URL监控:见{\color{blue}\href{http://blog.xiayf.cn/2013/10/12/url-monitoring-and-web-arch/}{文章}}说明
		\end{itemize}
		\item 相关技术:PHP、Redis、Memcached、Bootstrap、Highcharts、D3.js 等
	\end{itemize}
	
	\item {\color{blue}\href{http://blog.xiayf.cn/2013/10/16/high-availability-load-balancer-and-dns/}{高可用容灾方案 \tikz \draw[->, thick] (0,0) cos(0.1,0.2);}}
	\begin{itemize}
		\item 简介:主要针对易迅IDC与腾讯IDC之间的互通,提供高可用容灾方案
		\item 个人职责:
		\begin{itemize}
			\item 基于HAProxy和Keepalived提供网络双链路的负载均衡与互备
			\item 基于BIND为IDC内部搭建缓存DNS
			\item 使用Go语言实现HAProxy负载均衡任务管理系统 {\color{blue}\href{https://github.com/youngsterxyf/haproxyconsole}{HAProxyConsole \tikz \draw[->, thick] (0,0) cos(0.1,0.2);}}
		\end{itemize}
		\item 说明:该项目由个人独立完成;HAProxyConsole在腾讯其他部门也得到应用。
	\end{itemize}
	
	\item {\color{blue}\href{http://blog.xiayf.cn/2013/11/29/inner_warehouse_monitor_system/}{易迅全国仓库作业机器监控系统 \tikz \draw[->, thick] (0,0) cos(0.1,0.2);}}
	\begin{itemize}
		\item 简介:针对易迅业务全国仓库内的普通作业PC,采集、存储、展示CPU使用率、内存使用率、网卡流量、开关机状态等指标数据,以辅助机器故障分析
		\item 个人职责:系统的总体设计,技术选型,NSQ客户端实现(包含数据存储),数据的展示 等
		\item 相关技术:Go、Beego、NSQ、Saltstack 等
	\end{itemize}
	
	\item {\color{blue}\href{http://blog.xiayf.cn/2013/10/15/standardization-operation-development/}{运营开发规范化 \tikz \draw[->, thick] (0,0) cos(0.1,0.2);}}
    \begin{itemize}
    	\item 简介:针对团队协作存在的问题,定制一整套开发流程规范,以及规范的宣导
    	\item 相关技术:Linux、Gitlab、Git、Python、PHP、Go
    \end{itemize}

	\item 服务器日常运维工作及运维工具类开发
\end{itemize}

\cventry{2012.05 -- 2012.12}{Web开发实习生}{谷歌-企业社会责任部}{上海}{}{}
\begin{itemize}
	\item 工作简介:负责开发维护{\color{blue}\href{www.17gong1.com}{一起公益网}}、{\color{blue}\href{www.gong1pin.com}{公益品网}} 等
	\item 相关技术:CentOS、Nginx、PHP、MySQL、SVN、jQuery等
\end{itemize}


\section{个人项目}

\begin{itemize}
\item {\color{blue}\href{https://github.com/youngsterxyf/fuse}{支持多平台的Git webhook服务 \tikz \draw[->, thick] (0,0) cos(0.1,0.2);}}
	\begin{itemize}
		\item 简介:针对Git多分支工作流模型,以插件化方式实现多平台Webhook支持,配置灵活
		\item 技术点:Go、Martini、FlatUI等
	\end{itemize}
\item {\color{blue}\href{https://github.com/youngsterxyf/feed-world}{Feed聚合服务 \tikz \draw[->, thick] (0,0) cos(0.1,0.2);}}
	\begin{itemize}
		\item 技术点:Slim、Vue.js、Github及微博OAuth2登陆
	\end{itemize}
\item {\color{blue}\href{https://github.com/youngsterxyf/blink}{Linux服务器单机监控 \tikz \draw[->, thick] (0,0) cos(0.1,0.2);}}
	\begin{itemize}
		\item 简介:针对单机Linux服务器,采集CPU、内存、磁盘指标数据,定期绘制成图表发送到指定邮箱
		\item 技术点:Python、psutil、Pandas、vincent、vega 等
	\end{itemize}
\item {\color{blue}\href{https://github.com/youngsterxyf/Baidu_Music_Downloader}{百度音乐下载器 \tikz \draw[->, thick] (0,0) cos(0.1,0.2);}}
\begin{itemize}
	\item 简介:按歌名下载、按歌手名批量下载专辑或单曲、多歌手并发批量下载专辑或单曲
	\item 技术点:Python、Requests、BeautifulSoup、Gevent
\end{itemize}
\item {\color{blue}\href{http://blog.xiayf.cn/sphinx/work_note}{技术学习笔记 \tikz \draw[->, thick] (0,0) cos(0.1,0.2);}}
    \begin{itemize}
    \item 主要分为架构、开发、运维、专业基础四个部分,记录技术学习笔记、技术资源聚合
    \end{itemize}
\item {\color{blue}\href{http://cn.python-requests.org/zh_CN/latest/}{Requests文档中文翻译 \tikz \draw[->, thick] (0,0) cos(0.1,0.2);}}
    \begin{itemize}
    \item 完成大部分内容的翻译工作
    \end{itemize}
\end{itemize}


\section{教育背景}
\cventry{2010.9 -- 2013.4}{硕士}{上海交通大学 - 软件学院}{上海}{}{}  % 第3到第6编码可留白
\cventry{2006.9 -- 2010.7}{学士}{上海大学 - 计算机工程与科学学院}{上海}{}{}


\section{语言技能}

\cvitemwithcomment{英语}{CET-6}{读、写 - 流畅,听、说 - 一般}
\begin{itemize}
\item 经常翻译优秀技术文章(见{\color{blue}\href{http://blog.xiayf.cn}{个人博客}})
\item 每天阅读Hacker News,经常查询Stack Overflow,浏览High Scalability、Quora等网站
\end{itemize}


\section{个人兴趣}

\cvitem{技术分享}{\small 个人分享列表见 {\color{blue} \href{http://blog.xiayf.cn/pages/tech-share.html}{这里}};发起并组织 {\color{blue}\href{http://happytechgroup.github.io/}{众成技术聚乐部}}}
\cvitem{技术翻译}{\small 业余时间热衷于翻译自认为好的英文技术文章}
\cvitem{阅读}{\small 三日不读书便觉面目可憎。阅读甚广,尤其偏爱文史哲}

\clearpage\end{CJK}
\end{document}

%% 文件结尾 `template-zh.tex'.
